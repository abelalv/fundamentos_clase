%%% Template originaly created by Karol Kozioł (mail@karol-koziol.net) and modified for ShareLaTeX use

\documentclass[a4paper,11pt]{article}

\usepackage[T1]{fontenc}

\usepackage[utf8]{inputenc}
\usepackage{graphicx}
\usepackage{xcolor}
\usepackage{float}
\usepackage{tgtermes}
\usepackage{subcaption}
\usepackage[
pdftitle={MMAF}, 
pdfauthor={Pontificia Universidad Javeriana Cali},
colorlinks=true,linkcolor=blue,urlcolor=blue,citecolor=blue,bookmarks=true,
bookmarksopenlevel=2]{hyperref}
\usepackage{amsmath,amssymb,amsthm,textcomp}
\usepackage{enumerate}
\usepackage{multicol}
\usepackage{tikz}
\usepackage{booktabs}
\usepackage{geometry}
\geometry{
	left=25mm,
	right=25mm,
	bindingoffset=0mm,
	top=20mm,
	bottom=20mm
}


\linespread{1.3}

\newcommand{\linia}{\rule{\linewidth}{0.5pt}}

% custom theorems if needed
\newtheoremstyle{mytheor}
{1ex}{1ex}{\normalfont}{0pt}{\scshape}{.}{1ex}
{{\thmname{#1 }}{\thmnumber{#2}}{\thmnote{ (#3)}}}

\theoremstyle{mytheor}
\newtheorem{defi}{Definition}

% my own titles
\makeatletter
\renewcommand{\maketitle}{
	\begin{center}
		\vspace{2ex}
		{\huge \textsc{\@title}}				
		\vspace{1ex}
		\\
		\linia\\
		\@author \hfill \@date
		\vspace{4ex}
	\end{center}
}
\makeatother
%%%

% custom footers and headers
\usepackage{fancyhdr,lastpage}
\pagestyle{fancy}
\lhead{}
\chead{}
\rhead{}
\lfoot{Taller}
\cfoot{}
\rfoot{Page \thepage\ /\ \pageref*{LastPage}}
\renewcommand{\headrulewidth}{0pt}
\renewcommand{\footrulewidth}{0pt}
%

%%%----------%%%----------%%%----------%%%----------%%%

\begin{document}
	
	\title{Taller  MMAF}	
	\author{Pontificia Universidad Javeriana- Cali}
	
	%\date{01/01/2014}
	
	\maketitle
	
	\begin{enumerate}
		
		
		\item
Simplifica cada una de las siguientes expresiones usando identidades trigonométricas de ángulos dobles y suma de ángulos.
\begin{enumerate}
    \item \( \sin(2\theta) + \cos(2\theta) \)
    \item \( \tan(2\alpha) \)
    \item \( \cos(45^\circ + 30^\circ) \)
    \item \( \sin(60^\circ - 30^\circ) \)
\end{enumerate}

\item 
Encuentra el valor exacto de las siguientes expresiones utilizando funciones trigonométricas inversas.

\begin{table}[H]
    \centering
    \begin{tabular}{|c|c|c|c|c|}
    \hline
    \textbf{Expresión} & $0$ & $\pi/2$ & $\pi/3$ & $\pi/6$ \\ \hline
    $\cos$ & 1 & 0 & $\frac{1}{2}$ & $\frac{\sqrt{3}}{2}$ \\ \hline
    $\sin$ & 0 & 1 & $\frac{\sqrt{3}}{2}$ & $\frac{1}{2}$ \\ \hline
    $\tan$ & 0 & undefined & $\sqrt{3}$ & $\frac{1}{\sqrt{3}}$ \\ \hline
    \end{tabular}
    \caption{Valores de funciones trigonométricas para ángulos comunes}
    \label{tab:trig_values}
    \end{table}

\begin{enumerate}
    \item \( \arcsin(\sin(3\pi/4)) \)
    \item \( \arctan(\tan(-\pi/3)) \)
    \item \( \arccos(\cos(\pi)) \)
    \item \( \tan(\arctan(3) + \arctan(2)) \)
\end{enumerate}

\item
Determina si las siguientes ecuaciones son identidades trigonométricas.
\begin{enumerate}
    \item \( \sin^2(x) + \cos^2(x) = 1 \)
    \item \( 1 + \tan^2(x) = \sec^2(x) \)
    \item \( \sin(x + y) = \sin(x)\cos(y) + \cos(x)\sin(y) \)
    \item \( \cos(x + y) = \cos(x)\cos(y) - \sin(x)\sin(y) \)
\end{enumerate}

\item
Encuentra el valor exacto de las siguientes expresiones si \(\sin \theta = 3/5\) y \(\theta\) está en el segundo cuadrante.
\begin{enumerate}
    \item \( \cos(2\theta) \)
    \item \( \sin(2\theta) \)
    \item \( \tan(2\theta) \)
    \item \( \cot(2\theta) \)
\end{enumerate}

\item 

Dos fuerzas actúan sobre un objeto en un punto de origen. La fuerza \( F_1 = 80 \, \text{N} \) actúa en dirección de \( 30^\circ \) respecto al eje horizontal, y la fuerza \( F_2 = 60 \, \text{N} \) actúa en dirección de \( 120^\circ \) respecto al mismo eje.
\begin{enumerate}
    \item Descompón las fuerzas \( F_1 \) y \( F_2 \) en sus componentes horizontales y verticales.
    \item Calcula la fuerza resultante \( F_R \) sumando las componentes horizontales y verticales de \( F_1 \) y \( F_2 \).
    \item Encuentra el ángulo \( \theta \) que forma la fuerza resultante con el eje horizontal.
    \item Verifica usando identidades de ángulo doble que el ángulo resultante \( \theta \) puede expresarse mediante la combinación de las direcciones de \( F_1 \) y \( F_2 \).
    \end{enumerate}
 
\item
Modifique el código usado en el cuaderno 8 en colab y visualice las siguientes sucesiones y determine
a partir de la gráfica, si la sucesión converge o diverge.

\begin{enumerate}
    \item \( a_n = \dfrac{1}{n^2} \)
    \item \( a_n = (-1)^n\dfrac{1}{n} \)
    \item $a_n=\sin\left(\frac{1}{n}\right)$
    \item $a_n=\cos(n^2)$
\end{enumerate}

\item 
Determine la convergencia o divergencia de las siguientes sucesiones
\begin{enumerate}
    \item $\dfrac{3^n}{2^n+1}$
    \item $\dfrac{(3n-1)(n+2)}{(n+3)(n-5)}$
    \item $\dfrac{n}{4n^2+25}$
    \item $\sqrt{2n^2+5}-n$
\end{enumerate}

\item       
Suponga qque se define una sucesión de forma recurrente de la forma
$$a_{n+1}=\dfrac{a_n}{2}+\dfrac{1}{2}$$
 con $a_1=0$. Calcule según esta recurrencia $a_2,a_3,a_4,a_5$. Suponga que $\lim_{n \to \infty} a_n=L$. Cual
 es el valor de $L$?
\end{enumerate}
\end{document}

